\begin{titlepage}
%\voffset-40mm
\begin{spacing}{1.0}
	
\begin{tabularx}{\textwidth}{LYR}
	\includegraphics[height=1.8cm, keepaspectratio,]{logo_uniAugsburg.png} &
	\includegraphics[height=1.6cm, keepaspectratio,]{logo_AnalyticsAndOptimization.jpg} &
	\includegraphics[height=1.8cm, keepaspectratio,]{logo_fim_de_rot.png}
\end{tabularx}
\vspace*{1.2cm}
\begin{center}
	{\huge \textbf{Reinforcement Learning and Artificial Intelligence\\
	in the context of revenue management\\}} 
\vspace*{1.2cm}
{\Large Freie wissenschaftliche Arbeit \\
zur Erlangung des akademischen Grades\\
\enquote{Master of Science}\\
Studiengang: Finanz- und Informationsmanagement
} 
\\
\vspace*{1cm}
{\Large \textbf{an der\\
Wirtschaftswissenschaftlichen Fakultät\\
der Universität Augsburg\\
}}
\vspace*{1cm}
{\Large – Lehrstuhl für Analytics \& Optimization –\\}
\vspace*{1cm}
{
\begin{tabular}{ll}
Eingereicht bei:& Prof. Dr. Robert Klein\\
Betreuer:& 			Dr. Sebastian Koch\\
Vorgelegt von:&	Stefan Glogger\\
Adresse:&			tbd\\
Matrikel-Nr.:&		tbd\\
E-Mail:&				\href{mailto:stefan.glogger@student.uni-augsburg.de}{stefan.glogger@student.uni-augsburg.de}\\
Datum:&				\today
\end{tabular}
}
\end{center}

\end{spacing}

\end{titlepage}
