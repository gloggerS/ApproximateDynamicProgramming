\chapter{Main chapter}

\section{Ideas}
\begin{enumerate}
	\item capacity consumption $a \in \mathbb{N}_0$ statt $a\in \{0,1\}$
\end{enumerate}

\section{The problem}

Here, we want to lay out the classical revenue management problem an the \enquote{brute force} approach to solve. 

Consider a firm that produces products $j = 1, \dots, n$ with revenues $\boldsymbol{r} = (r_1, \dots, r_n)^T$. Resources $h = 1, \dots, m$ are used for production. In order to produce one unit of product $j$, resources $\boldsymbol{a}_j = (a_{1j}, \dots, a_{mj})^T$ are necessary, with $a_{hj}=1$ if resource $h$ is needed for production of product $j$ and $a_{hj} = 0$ otherwise. Initially, the capacity is described by $\boldsymbol{c}^0 = (c_1^0, \dots, c_m^0)^T$. 

The booking horizon is modelled by sufficiently small time periods $t = 0, \dots, T-1$, such that in each time period at most one customer arrives. This customer also purchases at most one product. If product $j$ is purchased at time $t$, the capacity reduces to $\boldsymbol{c}^{t+1} = \boldsymbol{c}^t - \boldsymbol{a}_j$. Time moves forward, such that the last selling might occur at time $T-1$.

% time index always at the top
The firm wants to increase the value of the products sold and has flexibility in the sets offered. Thus, the decision variables at each time point $t$ are given by $\boldsymbol{x}^t = (x^t_1, \dots, x^t_n)^T$ with $x^t_j = 1$ if product $j$ is offered at time $t$ and $x^t_j = 0$ otherwise.

One popular method of describing the probabilities of purchases is to have each customer belonging to one customer segment $l = 1, \dots L$, each of which following a multinomial logit model (MNL\nomenclature{MNL}{multinomial logit model}). A customer of segment $l$ arrives with probability $\lambda_i$. His preference weights are given by $\boldsymbol{u}_l = (u_{l1}, \dots, u_{ln})^T$ and no purchase preference of $u_{l0}$. Note: $u_{lj} > 0$ if consumer of segment $l$ might purchase product $j$ (the higher, the more interested) and $u_{lj} = 0$ if customer is not interested in product. The probability of purchasing product $j$ when set $\boldsymbol{x}$ is offered is given by $p_{lj}(\boldsymbol{x}) = \frac{u_{lj}x_j}{u_{l0} + \sum_{p\in[n]} u_{lp}x_p}$ and the no-purchase probability is given by $p_{l0}(\boldsymbol{x}) = 1 - \sum_{p\in[n]}p_{lp}$. Together with the uncertainty of which customer segment arrives (if any), we end up at a purchase probability for product $j$ given $\boldsymbol{x}$ of $p_J(\boldsymbol{x}) = \sum_{l \in [L]} p_{lj}(\boldsymbol{x})$.

\subsection{Examples}

\subsubsection{Single-leg flight example}

For reasons of comparability, we use the same example as in \cite{Koch.2017}. An airlines offers four products with revenues $\boldsymbol{r} = (1000, 800, 600, 400)^T$ over $T = 400$ periods. Only one customer segment exists with arrival probability of $\lambda = 0.5$ and preference weights $\boldsymbol{u} = (0.4, 0.8, 1.2, 1.6)^T$. Different network loads can be analyzed by varying initial capacity $c^0 \in \{40, 60, \dots, 120\}$ and varying no-purchase preference weights $u_0 \in \{1,2,3\}$.

\section{Implementation}

Here, we present the results of the exact calculation for the single leg flight example.

Storage folder: \texttt{"C:/Users/Stefan/LRZ Sync+Share/Masterarbeit-Klein/Code/Results/singleLegFlight-True-DP-190611-0917"}

Log:

\verbatiminput{"C:/Users/Stefan/LRZ Sync+Share/Masterarbeit-Klein/Code/Results/singleLegFlight-True-DP-190611-0917/0_logging.txt"}

Results:

\input{"C:/Users/Stefan/LRZ Sync+Share/Masterarbeit-Klein/Code/Results/singleLegFlight-True-DP-190611-0917/erg_paper.txt"}

\section{Approximate Dynamic Programming}

\subsection{Greedy Heuristic to determine offerset}

The following algorithm is based on the ideas of the greedy heuristic for the column generation subproblem outlined in \cite{Bront.2009}.

Our goal is to determine a reasonable set of products to offer in a fast manner. Thus, we use a heuristic and cut down the amount of products to consider as fast as possible.

