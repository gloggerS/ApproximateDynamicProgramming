\chapter{Main chapter}

\section{Notation}
\begin{enumerate}
	\item time index always at the top
	\item index for products always at bottom
\end{enumerate}



\subsection{Examples}

\subsubsection{Single-leg flight example}

For reasons of comparability, we use the same example as in \cite{Koch.2017}. An airline offers four products with revenues $\boldsymbol{r} = (1000, 800, 600, 400)^T$ over $T = 400$ periods. Only one customer segment exists with arrival probability of $\lambda = 0.5$ and preference weights $\boldsymbol{u} = (0.4, 0.8, 1.2, 1.6)^T$. Different network loads can be analyzed by varying initial capacity $c^0 \in \{40, 60, \dots, 120\}$ and varying no-purchase preference weights $u_0 \in \{1,2,3\}$.

\section{Implementation}

Here, we present the results of the exact calculation for the single leg flight example.

Storage folder: 

\texttt{"C:/Users/Stefan/LRZ Sync+Share/Masterarbeit-Klein/Code/Results/singleLegFlight-True-DP-190611-0917"}

Log:

\verbatiminput{"C:/Users/Stefan/LRZ Sync+Share/Masterarbeit-Klein/Code/Results/singleLegFlight-True-DP-190611-0917/0_logging.txt"}

Results:

\input{"C:/Users/Stefan/LRZ Sync+Share/Masterarbeit-Klein/Code/Results/singleLegFlight-True-DP-190611-0917/erg_paper.txt"}

Furthermore, we want to visualize the value function and its approximations. We use the example of $c^0 = 60$ and $u_0 = 1$. 

\subsection{Network flight example}

