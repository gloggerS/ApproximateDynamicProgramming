% based on scrbook
% finaler Ablauf: übersetzen, Tools -> Benutzer -> MakeNomenclature, übersetzen
\documentclass[
fontsize=12pt, %font size
paper=a4, %paper format
headsepline, %separating line for header
chapterprefix, % produce "Chapter ..."
numbers=noenddot, % no full-stop after last number
listof=totoc, %Entry in table of contents for list of figures/tables
index=totoc, %Entry in table of contents for index
bibliography=totoc, %Entry in table of contents for bibliography
%BCOR=5mm,%Binding correction, ensures sufficient space for binding
%todo change this to final to switch off for expamle the showlabels
%todo manually put 'disable' in front of usepackage{todonotes}
%final,%
]%
{scrbook}%

\KOMAoptions{twoside=false}




\usepackage{geometry}
\geometry{
a4paper,
total={150mm,205mm},
left=15mm,
right=50mm,
}

% 

% --------- Packages ----------
%\NeedsTeXFormat{LaTeX2e}
%\usepackage[dvips]{epsfig}
%\usepackage[latin1]{inputenc}
% TODO Argument "english" checken
\usepackage[english]{babel} 
%\usepackage[ngerman]{babel} 
\usepackage{a4wide}
%\usepackage{psfig}
\usepackage{fancyhdr}
\usepackage{latexsym}
\usepackage{enumerate}
\usepackage{paralist}
\usepackage{float}
\usepackage{verbatim}  % verbatiminput
\usepackage{caption}
\usepackage{floatflt}
\usepackage{afterpage}
\usepackage{graphicx}
\usepackage[space]{grffile}  % handles space in file name
\usepackage{moreverb}     
%\usepackage[]{mcode}
\usepackage{bbm}

% spacing of items: noitemsep 
\usepackage{enumitem}


% \usepackage{tabulary} % to set dith to side width and use left/right alignment
%\usepackage{showframe} % show margin of page


\usepackage[intoc]{nomencl}
\makenomenclature

% math
\usepackage{mathtools} % successor of \usepackage{amsmath}
\usepackage{amssymb,amsfonts}
\setcounter{MaxMatrixCols}{10} % leftover from Marcos
\usepackage{cases}

% Pseudo Code
\usepackage{algorithm}
\usepackage[noend]{algpseudocode}

%% bibliography
%%\usepackage[style=authoryear, backend=biber]{biblatex}
%%%\usepackage{biblatex}
%%\bibliography{bib}
%\usepackage[round, sort, comma]{natbib}
%\bibliographystyle{apalike}


% fraction for floats
\renewcommand{\floatpagefraction}{0.8}
%\renewcommand{\textfraction}{0.15}


% prevent floats from floating too much around (stay in one section)
%\usepackage[section]{placeins}

% adjust font of abbreviations
\usepackage{helvet}
\renewcommand{\nomlabel}[1]{{\fontfamily{phv}\selectfont \textbf{#1}}} %bold Helvetica



% ---------------------------------------------------------------------------------------------
% OWN PACKAGES 
% ---------------------------------------------------------------------------------------------

% tables
\usepackage{booktabs}
\usepackage{longtable}
\usepackage{tabularx} % especially for the title page
\newcolumntype{L}{>{\raggedright\arraybackslash}X}
\newcolumntype{Y}{>{\centering\arraybackslash}X}
\newcolumntype{R}{>{\raggedleft\arraybackslash}X}
\newcolumntype{x}{>{$}l<{$}} % math-mode version of "l" column type
\newcolumntype{y}{>{$}c<{$}} % math-mode version of "c" column type
\newcolumntype{z}{>{$}r<{$}} % math-mode version of "r" column type

\usepackage{multicol}



% language
\usepackage[utf8]{inputenc} % to type ä,ö,ü,...


% fast text
\usepackage{lipsum}

% bibliography
\usepackage[round, sort, comma]{natbib}
\bibliographystyle{apalike}

\usepackage{color}
\usepackage[dvipsnames]{xcolor}


% quotes
\usepackage{csquotes}
\usepackage[%
%	allbordercolors=black,
urlcolor=blue,
linkcolor=blue,
citecolor=.
]{hyperref}
\usepackage{cleveref}


% to show the assigned labels
% has to be included after the packages amsmath and hyperref
\usepackage{showlabels}

% todo
\usepackage%
[textwidth=40mm,
draft,%
%disable%
]%
{todonotes}


\usepackage{float}

%bold math symbols
\usepackage{bm}

% multiple figures
\usepackage{subcaption}


\usepackage[onehalfspacing]{setspace}

\usepackage{pdfpages}
% switched to .tex as then commands are directly included in AutoFill

% own package
% https://www.sharelatex.com/learn/Writing_your_own_package#Options

\NeedsTeXFormat{LaTeX2e}
\ProvidesPackage{z-style}


% colours
\RequirePackage[dvipsnames]{xcolor}
\definecolor{colStefan}{HTML}{e68a00}
\definecolor{colIdea}{HTML}{00cc00}

\RequirePackage{xstring} % for if statements


%todoNotes
\newcommand{\todoBib}[1]{\todo[color=green!40]{#1}}
%\newcommand{\todoRed}[2][noinline]{\todo[color=red!40, #1]{#2}}
\newcommand{\todoRed}[1]{\todo[inline, color=red!40]{#1}}
\newcommand{\todoMinor}[1]{\todo[inline, color=orange!20]{#1}}
\newcommand{\todoCurrentWork}[1]{\todo[inline, color=red!60]{CURRENTLY WORKING HERE: #1}\vspace*{5cm}}
\newcommand{\todoRequirememnt}[1]{\todo[color=cyan!10]{#1}}

% nice graphic for network example
\usetikzlibrary{arrows.meta,positioning}



% nicer coloured box
% https://tex.stackexchange.com/questions/66154/how-to-construct-a-coloured-box-with-rounded-corners
\usepackage[most]{tcolorbox}
\RequirePackage{tcolorbox}
\newtcolorbox{boxStefan}{colback=colStefan!5!white,colframe=colStefan!90}
\newtcolorbox{boxIdea}{colback=colIdea!5!white,colframe=colIdea!90}


% some often used typography
\newcommand{\eg}{\mbox{e.\,g.}\xspace}
\newcommand{\ie}{\mbox{i.\,e.}\xspace}
\newcommand{\wlogMath}{\mbox{w.\,l.\,o.\,g.}\xspace}

% ------------------------------------------------------------------------------
% Theorem-Umgebungen
\theoremnumbering{arabic}%

% first block

\theoremheaderfont{\bfseries}%
%\theorembodyfont{\upshape}%
%\theoremseparator{}%
\theorembodyfont{\normalfont}%
\theoremseparator{.}

\newtheorem{theorem}{Theorem}%
\crefname{theorem}{Theorem}{Theorems}
\Crefname{theorem}{Theorem}{Theorems}
\newtheorem{lemma}[theorem]{Lemma}%
\crefname{lemma}{Lemma}{Lemmas}
\Crefname{lemma}{Lemma}{Lemmas}
\newtheorem{definition}[theorem]{Definition}%
\crefname{definition}{Definition}{Definitions}
\Crefname{definition}{Definition}{Definitions}
\newtheorem{remark}[theorem]{Remark}%
\crefname{remark}{Remark}{Remarks}
\Crefname{remark}{Remark}{Remarks}

% second block

\theoremstyle{nonumberplain}%
\theoremheaderfont{\bfseries}%
\theorembodyfont{\normalfont}%
\theoremseparator{.}
\theoremsymbol{\ensuremath{\Box}}% ensuremath -> sowohl im Text- als auch im Mathemodus wird das gewünschte Zeichen gesetzt

\newtheorem{proof}{Proof}%
\crefname{proof}{Proof}{Proofs}
\Crefname{proof}{Proof}{Proofs}
% ------------------------------------------------------------------------------
%\newenvironment{proof}[1][Proof]{\noindent\textbf{#1.} }{\ \rule{0.5em}{0.5em}}
%\textwidth =170mm
%\textheight=205mm
%\oddsidemargin=-8mm
%\evensidemargin=-8mm
\DeclareMathOperator*{\argmax}{arg\,max}
\DeclareMathOperator*{\argmin}{arg\,min}


\begin{document}

\chapter{Organization}

\section{List of ToDos}

\todo{Layout: Zweiseitiges Layout (Druck)}
\todo{Absprache: Schwarz Weiß Version zusätzlich zu farbiger Druckversion?}
\listoftodos


\newpage
\section{Abzugeben}
\begin{enumerate}[noitemsep]
	\item Lehrstuhl Prof. Klein
	\begin{enumerate}[noitemsep]
		\item 2 ausgedruckte Versionen der Masterarbeit (je eine fuer Prof. Klein und Sebastian bei Melanie, und eine bei FIM)
		\item 1 USB-Stick
		\begin{enumerate}[noitemsep]
			\item PDF-Version der Masterarbeit
			\item LaTeX Code der Masterarbeit
			\item Python Code der Implementierung
			\item Grafiken und sonstige erzeugte Zwischenergebnisse
		\end{enumerate}
	\end{enumerate}
	\item FIM
	\begin{enumerate}[noitemsep]
		\item 1 ausgedruckte Version der Masterarbeit
		\item Bestaetigung zum Bestehen der Mastererabeit (ausgestellt von Prof. Klein)
		\item restliches Paket fuer Bestehen des Masterstudiums
	\end{enumerate}
	\item Persönlich
	\begin{enumerate}[noitemsep]
		\item Code auch in öffentlichem GitHub
		\item Zertifikat \enquote{Data Scientist with Python} vom DataCamp Career Track 
	\end{enumerate}
\end{enumerate}

Was ich erreichen mag:
\begin{enumerate}[noitemsep]
	\item Saubere Masterarbeit (das Dokument), v.a. saubere Ausarbeitung des Inhaltes (Wirtschafts-Komponente von FIM), aber auch \LaTeX Fähigkeiten aufgebessert und tikz Fähigkeiten verbessert
	\item Sauberer Code, der den aktuellen Programmierstandards entspricht (Informatik-Komponente von FIM)
	\item Paar kleinere Beweise dazu und saubere Verwendung mathematischer Terminologie, um auch mathematische Komponente von FIM (bzw. meines Wissens) zu zeigen
	\item Auch Fachliche Inhalte von FIM ab
	\begin{enumerate}[noitemsep]
		\item Optimierung: Modellierung, Lineares Programm (bzw. Programme) aufstellen, LP in Software implementieren, Ergebnisse visualisieren, Dualitätstheorie anwenden, Duales Programm aufstellen
		\item Wirtschaft: Nutzenfunktion, Komplementäre Güter, Substitute
		\item Wahrscheinlichkeitstheorie: Wahrscheinlichkeitsraum, Kolmogorov-Axiome, Zufallsvariable, Markov-Kette, 
		\item Statistik: Test aufstellen, Test implementieren, Testergebnisse darstellen
	\end{enumerate}
	\item DataCamp Career Track \enquote{Data Scientist with Python} komplett durch (100 Stunden, 26 Kurse, enthält zahlreiche Kurse auch zu Neuronalen Netzen und Machine Learning in Python)
\end{enumerate}

\newpage

\section{Abgesprochene Struktur}

Code
\begin{enumerate}
	\item NN soll laufen
\end{enumerate}

Inhaltlich

\begin{enumerate}
	\item Einleitung mit 
	\begin{enumerate}
		\item Literatur (erst später ausformulieren $\Rightarrow$ ADP in revenue management, gibt's schon NN)
		\item Problembeschreibung
		\item Notation
		\item evtl. Kundenwahlverhalten
	\end{enumerate} 
	\item Methoden mit nur mini Beispielen
	\begin{enumerate}
		\item Dynamic Programming
		\item CDLP
		\item API mit Miranda Bront Heuristik
	\end{enumerate}
	\item Neuronale Netzwerke
	\begin{enumerate}
		\item Grundlagen
		\item Anwendung
	\end{enumerate}
	\item Theorie zum Vergleich verschiedener Methoden (Statistik) $\Rightarrow$ wohl auch für Prof. Klein interessant
	\begin{enumerate}
		\item jeweils zwei miteinander
		\item alle gegen alle
	\end{enumerate}
	\item 1 Single Leg Beispiel
	\item 1 Multi Leg Beispiel
	\item Zusammenfassung
\end{enumerate}

Schmankerl

\begin{enumerate}
	\item Beweis zu exponential smoothing
	\item Laufzeitüberblick der verschiedenen Algorithmen (jeweils offline und online)
	\item Laufzeitanalyse verschiedene Implementierungen Dynamic Programming (Opt-Modell vs alles ausrechnen) $\Rightarrow$ so nicht machen, da \enquote{unfairer} Vergleich, weil immer Speicher vs. Rechenzeit
\end{enumerate}

\section{Introduction}
\begin{enumerate}[noitemsep]
	\item Entstehung Revenue Management
	\item klassische Annahme des independent demand greift zu kurz
	\item stattdessen Modellierung von Customer Behaviour
	\item Problem des Customer Control Based Revenue Managements
	\item Relevanz von Revenue Management
	\item Weitere Fragen, mit denen sich RM beschäftigt
	\item Ziel der Arbeit
	\item Kurzüberblick der Kapitel
\end{enumerate}

\section{Problem Description}
\begin{boxStefan}
	\begin{enumerate}
		\item time index always at the top
		\item index for products always at bottom
	\end{enumerate}
	
	Anständiges Symbolverzeichnis, z.B. mit glossaries \url{https://ctan.org/pkg/glossaries}
	
	Wertebereich der Variablen
\end{boxStefan}


\end{document}
