
\usepackage{geometry}
\geometry{
a4paper,
total={150mm,205mm},
left=15mm,
right=50mm,
}

% 

% --------- Packages ----------
%\NeedsTeXFormat{LaTeX2e}
%\usepackage[dvips]{epsfig}
%\usepackage[latin1]{inputenc}
% TODO Argument "english" checken
\usepackage[english]{babel} 
%\usepackage[ngerman]{babel} 
\usepackage{a4wide}
%\usepackage{psfig}
\usepackage{fancyhdr}
\usepackage{latexsym}
\usepackage{enumerate}
\usepackage{paralist}
\usepackage{float}
\usepackage{verbatim}  % verbatiminput
\usepackage{caption}
\usepackage{floatflt}
\usepackage{afterpage}
\usepackage{graphicx}
\usepackage[space]{grffile}  % handles space in file name
\usepackage{moreverb}     
%\usepackage[]{mcode}
\usepackage{bbm}

% ------------------------------------------------------------------------------
% To store values for the title page
% renewcommand falls Befehl schon existiert (von scrbook) - werde ich aber nicht in der bereits definierten Weise verwenden -> Überschreiben ok
\renewcommand{\title}[1]{\newcommand{\Title}{#1}}
% ------------------------------------------------------------------------------

% spacing of items: noitemsep 
\usepackage{enumitem}


% \usepackage{tabulary} % to set dith to side width and use left/right alignment
%\usepackage{showframe} % show margin of page


\usepackage[intoc]{nomencl}
\makenomenclature

% math
\usepackage{mathtools} % successor of \usepackage{amsmath}
\usepackage{amssymb,amsfonts}
\usepackage[hyperref, amsmath, thmmarks]{ntheorem}
\setcounter{MaxMatrixCols}{10} % leftover from Marcos
\usepackage{cases}

% absolute value and norm
\DeclarePairedDelimiter\abs{\lvert}{\rvert}%
\DeclarePairedDelimiter\norm{\lVert}{\rVert}%

% Swap the definition of \abs* and \norm*, so that \abs
% and \norm resizes the size of the brackets, and the 
% starred version does not.
\makeatletter
\let\oldabs\abs
\def\abs{\@ifstar{\oldabs}{\oldabs*}}
%
\let\oldnorm\norm
\def\norm{\@ifstar{\oldnorm}{\oldnorm*}}
\makeatother



% Pseudo Code
\usepackage{algorithm}
\usepackage[noend]{algpseudocode}

%% bibliography
%%\usepackage[style=authoryear, backend=biber]{biblatex}
%%%\usepackage{biblatex}
%%\bibliography{bib}
%\usepackage[round, sort, comma]{natbib}
%\bibliographystyle{apalike}


% fraction for floats
\renewcommand{\floatpagefraction}{0.8}
%\renewcommand{\textfraction}{0.15}


% prevent floats from floating too much around (stay in one section)
%\usepackage[section]{placeins}

% adjust font of abbreviations
\usepackage{helvet}
\renewcommand{\nomlabel}[1]{{\fontfamily{phv}\selectfont \textbf{#1}}} %bold Helvetica



% ---------------------------------------------------------------------------------------------
% OWN PACKAGES 
% ---------------------------------------------------------------------------------------------

% tables
\usepackage{booktabs}
\usepackage{longtable}
\usepackage{tabularx} % especially for the title page
\newcolumntype{L}{>{\raggedright\arraybackslash}X}
\newcolumntype{Y}{>{\centering\arraybackslash}X}
\newcolumntype{R}{>{\raggedleft\arraybackslash}X}
\newcolumntype{x}{>{$}l<{$}} % math-mode version of "l" column type
\newcolumntype{y}{>{$}c<{$}} % math-mode version of "c" column type
\newcolumntype{z}{>{$}r<{$}} % math-mode version of "r" column type
%%% for linebreak in cell (https://tex.stackexchange.com/questions/2441/how-to-add-a-forced-line-break-inside-a-table-cell#19678)
%Foo bar & \specialcell{Foo\\bar} & Foo bar \\    % vertically centered
%Foo bar & \specialcell[t]{Foo\\bar} & Foo bar \\ % aligned with top rule
%Foo bar & \specialcell[b]{Foo\\bar} & Foo bar \\ % aligned with bottom rule
\newcommand{\specialcell}[2][c]{%
	\begin{tabular}[#1]{@{}c@{}}#2\end{tabular}}
\usepackage{multicol}



% language
\usepackage[utf8]{inputenc} % to type ä,ö,ü,...


% fast text
\usepackage{lipsum}

% bibliography
\usepackage[round, sort, comma]{natbib}
\bibliographystyle{apalike}

\usepackage{color}
\usepackage[dvipsnames]{xcolor}


% quotes
% ------------------------------------------------------------------------------
% hyperref settings for PDF files
\usepackage[%
%	allbordercolors=black,
urlcolor=NavyBlue,
linkcolor=NavyBlue,
citecolor=.
%%% von Bachelorarbeit Mathe
%allbordercolors=black,%
%allcolors=black,%
%breaklinks=true,% allow links to break across lines
%colorlinks=false,%switch off colouring
%pdfpagelabels=true,% set PDF page labels (i.e. for Acrobat Reader), cf http://www.tex.ac.uk/cgi-bin/texfaq2html?label=pdfpagelabels (needed for index to work)
%plainpages=false,% Make page anchors use formatted form of page number, cf http://www.tex.ac.uk/cgi-bin/texfaq2html?label=pdfpagelabels (needed for index to work)
%hypertexnames=true,% use guessable names for links (needed for index to work)
%bookmarksopen=false,%
%bookmarksnumbered=false,%
%hyperfootnotes=false% incompatible with package footmisc
]%
{hyperref}%
% ------------------------------------------------------------------------------
\usepackage{csquotes}
\usepackage{cleveref}


% to show the assigned labels
% has to be included after the packages amsmath and hyperref
\usepackage{showlabels}

% todo
\usepackage%
[textwidth=40mm,
%draft,%
disable%
]%
{todonotes}


\usepackage{float}

%bold math symbols
\usepackage{bm}

% multiple figures - subfigure and subcaption are not compatible
%\usepackage{subfigure} 
\usepackage{subcaption}
\usepackage{framed} % um Box um subfigures zu malen

\usepackage{fancyvrb}

% redefine \VerbatimInput
\RecustomVerbatimCommand{\VerbatimInput}{VerbatimInput}%
{fontsize=\small,
	%
	frame=lines,  % top and bottom rule only
	framesep=2em, % separation between frame and text
	rulecolor=\color{Gray},
	%
	commandchars=\|, % escape character and argument delimiters for commands within the verbatim
	commentchar=*        % comment character
}



\usepackage[onehalfspacing]{setspace}

\usepackage{pdfpages}

% beautiful drawings
\usepackage{tikz}
\usetikzlibrary{snakes}

% brings the floating barrier \FloatBarrier
\usepackage{placeins}

% brings \xspace to correctly space abbreviations 
\usepackage{xspace}

% ==============================================================================
% eigens definierte Kommandos
% ==============================================================================

% in z-style.tex
