\chapter{Organization}

\section{List of ToDos}

\todo{Layout: Zweiseitiges Layout (Druck)}
\todo{Abgesprochen: Schwarz Weiß gedruckt würde genügen. Mache anständig.}
\listoftodos


\newpage
\section{Abzugeben}
\begin{enumerate}[noitemsep]
	\item Lehrstuhl Prof. Klein
	\begin{enumerate}[noitemsep]
		\item 2 ausgedruckte Versionen der Masterarbeit (je eine fuer Prof. Klein und Sebastian bei Melanie, und eine bei FIM)
		\item 1 USB-Stick
		\begin{enumerate}[noitemsep]
			\item PDF-Version der Masterarbeit
			\item LaTeX Code der Masterarbeit
			\item Python Code der Implementierung
			\item Grafiken und sonstige erzeugte Zwischenergebnisse
		\end{enumerate}
	\end{enumerate}
	\item FIM
	\begin{enumerate}[noitemsep]
		\item 1 ausgedruckte Version der Masterarbeit
		\item Bestaetigung zum Bestehen der Mastererabeit (ausgestellt von Prof. Klein)
		\item restliches Paket fuer Bestehen des Masterstudiums
	\end{enumerate}
	\item Persönlich
	\begin{enumerate}[noitemsep]
		\item Code auch in öffentlichem GitHub
		\item Zertifikat \enquote{Data Scientist with Python} vom DataCamp Career Track 
	\end{enumerate}
\end{enumerate}

Was ich erreichen mag:
\begin{enumerate}[noitemsep]
	\item Saubere Masterarbeit (das Dokument), v.a. saubere Ausarbeitung des Inhaltes (Wirtschafts-Komponente von FIM), aber auch \LaTeX Fähigkeiten aufgebessert und tikz Fähigkeiten verbessert
	\item Sauberer Code, der den aktuellen Programmierstandards entspricht (Informatik-Komponente von FIM)
	\item Paar kleinere Beweise dazu und saubere Verwendung mathematischer Terminologie, um auch mathematische Komponente von FIM (bzw. meines Wissens) zu zeigen
	\item Auch Fachliche Inhalte von FIM abgedeckt
	\begin{enumerate}[noitemsep]
		\item Optimierung: Modellierung, Lineares Programm (bzw. Programme) aufstellen, LP in Software implementieren, Ergebnisse visualisieren, Dualitätstheorie anwenden, Duales Programm aufstellen
		\item Wirtschaft: Nutzenfunktion, Komplementäre Güter, Substitute
		\item Wahrscheinlichkeitstheorie: Wahrscheinlichkeitsraum, Kolmogorov-Axiome, Zufallsvariable, Markov-Kette, 
		\item Statistik: Test aufstellen, Test implementieren, Testergebnisse darstellen
	\end{enumerate}
	\item DataCamp Career Track \enquote{Data Scientist with Python} komplett durch (100 Stunden, 26 Kurse, enthält zahlreiche Kurse auch zu Neuronalen Netzen und Machine Learning in Python)
	\item Saubere Notation (fußnotieren, falls und warum von Literaturvorgabe abgewichen)
\end{enumerate}

\newpage

\section{Struktur}

Inhaltlich

\begin{enumerate}
	\item Einleitung mit 
	\begin{enumerate}
		\item Literatur (erst später ausformulieren $\Rightarrow$ ADP in revenue management, gibt's schon NN)
		\item Problembeschreibung
		\item Notation
		\item evtl. Kundenwahlverhalten
	\end{enumerate} 
	\item Methoden mit nur mini Beispielen
	\begin{enumerate}
		\item Dynamic Programming
		\item CDLP
		\item API mit Miranda Bront Heuristik
	\end{enumerate}
	\item Neuronale Netzwerke
	\begin{enumerate}
		\item Grundlagen
		\item Anwendung
	\end{enumerate}
	\item Theorie zum Vergleich verschiedener Methoden (Statistik) $\Rightarrow$ wohl auch für Prof. Klein interessant
	\begin{enumerate}
		\item jeweils zwei miteinander
		\item alle gegen alle
	\end{enumerate}
	\item 1 Single Leg Beispiel
	\item 1 Multi Leg Beispiel
	\item Zusammenfassung
\end{enumerate}

Schmankerl

\begin{enumerate}
	\item Beweis zu exponential smoothing
	\item Laufzeitüberblick der verschiedenen Algorithmen (jeweils offline und online)
	\item Laufzeitanalyse verschiedene Implementierungen Dynamic Programming (Opt-Modell vs alles ausrechnen) $\Rightarrow$ so nicht machen, da \enquote{unfairer} Vergleich, weil immer Speicher vs. Rechenzeit
	\item farbige Box am Ende eines einleitenden Kapitels zu Algorithmen dazu, was genau jetzt verwendet wird, um Ergebnisse zu produzieren.
	\item Best Practices
	\begin{enumerate}
		\item Python Career Track in Data Science machen, das erleichtert das Arbeiten in Python
		\item Klare Ziele für den Tag stecken motiviert
		\item In Sprints arbeiten erhält die Motivation und schafft Zeit frei für Abwechslung
	\end{enumerate}
\end{enumerate}