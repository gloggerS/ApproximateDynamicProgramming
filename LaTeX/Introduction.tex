\chapter{Introduction}

\begin{enumerate}[noitemsep]
	\item Entstehung Revenue Management
	\item klassische Annahme des independent demand greift zu kurz
	\item stattdessen Modellierung von Customer Behaviour
	\item Problem des Customer Control Based Revenue Managements
	\item Relevanz von Revenue Management
	\item Weitere Fragen, mit denen sich RM beschäftigt
	\item Ziel der Arbeit
	\item Kurzüberblick der Kapitel
\end{enumerate}

The concept of revenue management originated as a result of the U.S. Airline Deregulation Act of 1978 \cite{Talluri.2005}. Prior to this, prices had been strictly controlled and standardized by the U.S. Civil Aviation Board (CAB\nomenclature{CAB}{U.S. Civil Aviation Board}). Afterwards, airlines were free to choose prices, change schedules and alter offered services at their will, without CAB approval. This deregulation resulted in the immediate appearance of low-cost/no-frill airlines. By operating on lower labour costs and less service on board, profitable fares up to $70 \%$ lower than the ones of major carriers could be offered and an market pressure on classical large airlines increased. They now faced the problem of which price to offer at any given circumstances. 

The traditional, rather simple assumption of independent demand didn't work. It expects demand to be independent of the current market conditions such as prices offered by competitors, (day-)time of departure, frequency of departures or brand image \cite{Talluri.2005}. Furthermore, low fare demand is generally expected to come before high fare demand (as business travellers prefer the flexibility to possibly change their schedule). The problem reduced to: How much capacity should be reserved for the high fare demand, that appears later? But as the purchase of a ticket clearly depends on the market situation (e.g. an exceptional Formula 1 race at one specific weekend) or the continuing expansion of low-cost airlines offering undifferentiated fare structures, this simple assumption of independent falls short in practice \cite{Bront.2009}.

As a result, academia shifted its focus to more sophisticated models building on customer choice behaviour. Demand is assumed to depend on the currently offered products, which might change depending on the circumstances. This assumption now allows to also account for buy-up (buying a higher fare when lower fares are closed) and buy-down (switching to a lower fare when discounts are available). The assumption of customer choice is then incorporated in dynamic pricing models, as e.g. studied by \cite{Gallego.1997}, \cite{Bitran.1998} or \cite{Feng.2000}.

The underlying problem is referred to as capacity control under customer choice behaviour and can be summarized as: Which products should be offered at any given circumstance, whereby circumstance is determined mainly by time to departure and remaining capacity. As many airlines operate so called hub-and-spoke networks, allowing them to offer service in many more markets than with point-to point connection, the problem becomes more challenging \cite{Talluri.2005}. This is incorporated by modelling each point-to-point connection (single leg) by one separate resource and each seat as one unit of capacity. Furthermore, different customers (e.g. business vs. leisure traveller) can be distinguished by modelling them as different customer segments. Thus, the central problem in capacity control can be stated as:

\begin{center}
	\emph{Given a network consisting of several flights (edges) connecting certain cities (vertices) with varying seat capacity (weight of edge) and given a certain time to departure, which combination of products shall be offered to the customers?}
\end{center} 

Note that revenue management is relevant in many fields, even though we introduced the problems purely in the airline setting. The introduction was done intentionally like this as revenue management is closely connected to the airline industry and it increases total profitability by $4$ to $5 \%$ of revenues \cite{Talluri.2005}\footnote{We want to point out that Soutwest Airlines is often mentioned as a counterexample. But even though its pricing structure is very simple, revenue management systems are still used \cite{Talluri.2005}.}. However, revenue management appears in any decision related to demand management. and \cite{Talluri.2005} structured those decisions in three basic categories:

\begin{itemize}
	\item Structural decisions: How the selling process is organized (e.g. negotiations among individuals, auctions with restricted access, publicly posted prices) or how additional terms are structured (e.g. volume discounts or refund options).
	\item Price decisions: How to set individual offer prices, reserve prices (in auctions) and posted prices or how to price over time.
	\item Quantity decisions: Whether an offer to buy should be accepted or rejected or when to withhold a product from the market in order to sell it at a later point in time.
\end{itemize}

With this general framework and the relevance of revenue management in mind, we can formulate the goal of this thesis. We want to explore different methods of \enquote{solving} the capacity control problem under choice behaviour in the single-leg and multi-leg setting by implementing the algorithms in Python, let them cope with exemplary test scenarios and compare their performance. Furthermore, new methods such as the usage of Neural Networks shall be discussed and evaluated. 

The remaining of the thesis is structured as follows. 
\Cref{ch:Literature} gives an overview of existing literature.
\Cref{ch:ProblemsMethods} describes the problem formally and introduces various solution methods.
\Cref{ch:ComparisonTheory} presents theory on how to compare different methods based on statistics.
\Cref{ch:Examples} evaluates the different methods in one single-leg and one multi-leg scenario.
Finally, \Cref{ch:conclusion} summarizes the thesis and presents an outlook for future research. The code of the implementation is given in \Cref{ch:code} but can also be found on \todo{LM: Referenz zum Code überprüfen}\todo{GitHub repository für Abgabe der Masterarbeit}.