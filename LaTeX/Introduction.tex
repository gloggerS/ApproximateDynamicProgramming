\chapter{Introduction}

This document presents my Master's Thesis for the M.\,Sc.\xspace Programme \emph{Finance and Information Management}, in which I explain, implement and compare performance of various techniques and algorithms in revenue management. This first chapter gives a short overview of the history of revenue management, introduces the problem to be discussed in the following chapters and puts the thesis in context of different fields of revenue management. It furthermore gives a short overview of the chapters to come and briefly elaborates on how I see this thesis fitting in the context of my study programmes.

The concept of revenue management originated as a result of the U.\,S.\xspace Airline Deregulation Act of 1978. Prior to this, prices had been strictly controlled and standardized by the U.\,S.\xspace Civil Aviation Board (CAB\nomenclature{CAB}{U.\,S.\xspace Civil Aviation Board}). Afterwards, airlines were free to choose prices, change schedules and alter offered services at their will, without CAB approval. This deregulation resulted in an immediate appearance of low-cost/no-frill airlines. By operating on lower labour costs and less service on board, profitable fares up to $70 \%$ lower than the ones of major carriers could be offered and market pressure on classical airlines increased. All airline companies now faced the problem of which price to offer at any given circumstances. 

The traditional, rather simple assumption of independent demand didn't work. It expects demand to be independent of the current market conditions such as prices offered by competitors, (day-)time of departure, frequency of departures, brand image or others, to be found in \eg \cite{Talluri.2005}. Furthermore, low fare demand is generally expected to come before high fare demand, as business travellers prefer the flexibility to possibly change their schedule. Thus, the problem reduced to: How much capacity should be reserved for the high fare demand, that appears later? But as the purchase of a ticket clearly depends on the market situation (\eg an exceptional Formula 1 race at one specific weekend) or the continuing expansion of low-cost airlines offering undifferentiated fare structures, this simple assumption of independent falls short in practice as also stated in \cite{Bront.2009}.

As a result, academia shifted focus to more sophisticated models building on customer choice behaviour. Demand is assumed to depend on the currently offered products, which might change depending on the circumstances. This assumption now allows to also account for buy-up (buying a higher fare when lower fares are closed) and buy-down (switching to a lower fare when discounts are available). The assumption of customer choice is then incorporated in dynamic pricing models, as \eg studied by \cite{Gallego.1997}, \cite{Bitran.1998} or \cite{Feng.2000}.

The underlying problem is referred to as \emph{capacity control under customer choice behaviour} and can be summarized as: Which products should be offered at any given circumstance? In the airline setting, circumstance is determined mainly by time to departure and remaining capacity. As many airlines operate so called \emph{hub-and-spoke networks}, allowing them to offer service in many more markets than with point-to point connection, the problem becomes more challenging as stated \eg in \cite{Talluri.2005}. This network structure can be incorporated by modelling each point-to-point connection (single leg) by one separate resource and each seat as one unit of capacity. Furthermore, different customers (\eg business vs. leisure traveller) can be distinguished by modelling them as different customer segments. Thus, the central problem in capacity control can be stated as:

\begin{center}
	\emph{In the setting of a network consisting of several flights (edges) connecting certain cities (vertices), when given a particular seat capacity (weight of edge) and a certain time to departure, which combination of products shall be offered to customers?}
\end{center} 

Note that revenue management is relevant in many fields, even though we introduced the problems purely in the airline setting. The introduction was done intentionally like this as revenue management is closely connected to the airline industry and it increases total profitability by $4$ to $5 \%$ of revenues \cite{Talluri.2005}\footnote{We want to point out that Soutwest Airlines is often mentioned as a counterexample. But even though its pricing structure is very simple, revenue management systems are still used as pointed out by \cite{Talluri.2005}.}. However, revenue management appears in any decision situation related to demand management. \cite{Talluri.2005} structured those decisions in three basic categories:

\begin{itemize}
	\item Structural decisions: How the selling process is organized (\eg negotiations among individuals, auctions with restricted access, publicly posted prices) or how additional terms are structured (\eg volume discounts or refund options).
	\item Price decisions: How to set individual offer prices, reserve prices (in auctions) and posted prices or how to price distinct products over time.
	\item Quantity decisions: Whether an offer to buy should be accepted or rejected or when to withhold a product from the market in order to sell it at a later point in time.
\end{itemize}

For this thesis, structural and price decisions have been made already: Prices are fixed in advance and posted publicly. The quantity decisions remains and we can now formulate the goal of this thesis. We want to explore different methods of \enquote{solving} the capacity control problem under choice behaviour in a single-leg and multi-leg setting by implementing the algorithms, let them cope with exemplary test scenarios and compare their performance. Furthermore, new methods such as the usage of Neural Networks shall be discussed and evaluated. \todo{NN ggf. anpassen}

The remaining of the thesis is structured as follows. 
\Cref{ch:Literature} gives an overview of existing literature.
\Cref{ch:ProblemsMethods} describes the problem formally and introduces various solution methods.
\Cref{ch:ComparisonTheory} presents theory on how to compare different methods based on statistics.
\Cref{ch:Examples} evaluates the different methods in one single-leg and one multi-leg scenario.
Finally, \Cref{ch:conclusion} summarizes the thesis and presents an outlook for future research. The code of the implementation is given in \Cref{ch:code} but can also be found on \todo{LM: Referenz zum Code überprüfen}\todo{GitHub repository für Abgabe der Masterarbeit}.

Briefly, I want to elaborate on how I see this thesis fitting into my study programmes and my personal goals going along with it. This thesis shall combine the programme Finance and Information Management, with its interdisciplinary components of Business, Mathematics and Informatics, together with my personal interests and professional working attitude. The optimization component is covered by modelling the situation, creating an optimization problem, implementing this in software, visualizing the results and applying duality theory. The business component can be found in concepts like utility function, complementary assets and substitutes. Basic probability theory is incorporated with probability spaces, Kolmogorov-axioms, random variables and markov chains. Concepts of statistics are included as statistical tests are created, implemented and results presented. The informatics component is covered by preparing code in the modern\footnote{Python is heavily used in the Data Science community.} programming language Python and following modern software standards, as introduced by \href{https://www.python.org/dev/peps/pep-0020/}{The Zen of Python}, the style guide \href{https://www.python.org/dev/peps/pep-0008/}{PEP 8} and reproducibility. The mathematical component can be found throughout the thesis by precise notation, reformulation of some concepts found in papers, and short mathematical components in footnotes or the appendix. The thesis should be a neat document, properly set in \LaTeX~and showing my further enhanced typographic and layout skills, also in TikZ. Overall, this thesis should present a concise overview of techniques currently applied in revenue management and can be used as a starting ground for future research in this field.