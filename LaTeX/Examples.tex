\chapter{Examples}\label{ch:Examples}

\section{Single-Leg Flight}

For reasons of comparability, we use the same example as in \cite{Koch.2017}. An airline offers four products with revenues $\boldsymbol{r} = (1000, 800, 600, 400)^T$ over $T = 400$ periods. Only one customer segment exists with arrival probability of $\lambda = 0.5$ and preference weights $\boldsymbol{u} = (0.4, 0.8, 1.2, 1.6)^T$. Different network loads can be analyzed by varying initial capacity $c^0 \in \{40, 60, \dots, 120\}$ and varying no-purchase preference weights $u_0 \in \{1,2,3\}$.

\subsection{Implementation}

Here, we present the results of the exact calculation for the single leg flight example.

Storage folder: 

\texttt{"C:/Users/Stefan/LRZ Sync+Share/Masterarbeit-Klein/Code/Results/singleLegFlight-True-DP-190611-0917"}

Log:

\verbatiminput{"C:/Users/Stefan/LRZ Sync+Share/Masterarbeit-Klein/Code/Results/singleLegFlight-True-DP-190611-0917/0_logging.txt"}

Results:

\input{"C:/Users/Stefan/LRZ Sync+Share/Masterarbeit-Klein/Code/Results/singleLegFlight-True-DP-190611-0917/erg_paper.txt"}

Furthermore, we want to visualize the value function and its approximations. We use the example of $c^0 = 60$ and $u_0 = 1$. 

\section{Multi-Leg Flight}

\section{Old-Example0}

One interesting result is that for exact reproduction of the results of Bront et al for their example0, the empty offerset has to be excluded from the optimization problem. Otherwise (including empty set) results in the same optimal value but via another combination of the decision variables.

\textit{Example 0} represents a small airline network. Three cities are interconnected by three flights (legs), with a capacity vector $\boldsymbol{c} = (10, 5, 5)^T$. The booking horizon consists of $T = 30$ periods and there are five customer segments with preferences as in \Cref{tb-Example0-Customers}

\begin{table}
	\caption{Segment definition for Example 0\label{tb-Example0-Customers}}
	\begin{tabular}{lccccc}
		\toprule
		Segment & $\lambda_l$ & Consideration set & Preference vector & No purchase preference & Description\\
		\midrule
		1 & $0.15$ & $\{1, 5\}$ & $(5, 8)$ & $2$ & Price sensitive, nonstop (A$\rightarrow$C)\\
		\bottomrule
	\end{tabular}
\end{table}

\begin{figure}
	\caption{Airline network for example 0. \label{fig-Example0}}
	\begin{tikzpicture}[
	mycircle/.style={
		circle,
		draw=black,
		fill=gray,
		fill opacity = 0.3,
		text opacity=1,
		inner sep=0pt,
		minimum size=20pt,
		font=\small},
	myarrow/.style={-Stealth},
	node distance=0.6cm and 1.2cm
	]
	\node[mycircle] (cB) {$B$};
	\node[mycircle,below left=of cB] (cA) {$A$};
	\node[mycircle,below right=of cB] (cC) {$C$};
	
	\foreach \i/\j/\txt/\p in {% start node/end node/text/position
		cA/cB/Leg 1/above,
		cA/cC/Leg 2/below,
		cB/cC/Leg 3/above}
	\draw [myarrow] (\i) -- node[sloped,font=\small,\p] {\txt} (\j);
	
	\end{tikzpicture}
\end{figure}

\begin{table}
	\caption{Airline network for example 0 (products). \label{tb-Example0-Products}}
	\begin{tabular}{yxz}
		\toprule
		\text{Product} & \text{Origin-destination} & \text{Fare}\\
		\midrule
		1 & A \rightarrow C & 1,200\\
		2 & A \rightarrow B \rightarrow C & 800\\
		3 & A \rightarrow B & 500\\
		4 & B \rightarrow C & 500\\
		5 & A \rightarrow C & 800\\
		6 & A \rightarrow B \rightarrow C & 500\\
		7 & A \rightarrow B & 300\\
		8 & B \rightarrow C & 300\\
		\bottomrule
	\end{tabular}
\end{table}