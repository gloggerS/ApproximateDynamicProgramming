% switched to .tex as then commands are directly included in AutoFill

% own package
% https://www.sharelatex.com/learn/Writing_your_own_package#Options

\NeedsTeXFormat{LaTeX2e}
\ProvidesPackage{z-style}


% colours
\RequirePackage[dvipsnames]{xcolor}
\definecolor{colStefan}{HTML}{e68a00}
\definecolor{colIdea}{HTML}{00cc00}

\RequirePackage{xstring} % for if statements


%todoNotes
\newcommand{\todoBib}[1]{\todo[color=green!40]{#1}}
%\newcommand{\todoRed}[2][noinline]{\todo[color=red!40, #1]{#2}}
\newcommand{\todoRed}[1]{\todo[inline, color=red!40]{#1}}
\newcommand{\todoMinor}[1]{\todo[inline, color=orange!20]{#1}}
\newcommand{\todoCurrentWork}[1]{\todo[inline, color=red!60]{CURRENTLY WORKING HERE: #1}\vspace*{5cm}}
\newcommand{\todoRequirememnt}[1]{\todo[color=cyan!10]{#1}}

% nice graphic for network example
\usetikzlibrary{arrows.meta,positioning}



% nicer coloured box
% https://tex.stackexchange.com/questions/66154/how-to-construct-a-coloured-box-with-rounded-corners
\usepackage[most]{tcolorbox}
\RequirePackage{tcolorbox}
\newtcolorbox{boxStefan}{colback=colStefan!5!white,colframe=colStefan!90}
\newtcolorbox{boxIdea}{colback=colIdea!5!white,colframe=colIdea!90}


% some often used typography
\newcommand{\eg}{\mbox{e.\,g.}\xspace}
\newcommand{\ie}{\mbox{i.\,e.}\xspace}
\newcommand{\wlogMath}{\mbox{w.\,l.\,o.\,g.}\xspace}

% ------------------------------------------------------------------------------
% Theorem-Umgebungen
\theoremnumbering{arabic}%

% first block

\theoremheaderfont{\bfseries}%
%\theorembodyfont{\upshape}%
%\theoremseparator{}%
\theorembodyfont{\normalfont}%
\theoremseparator{.}

\newtheorem{theorem}{Theorem}%
\crefname{theorem}{Theorem}{Theorems}
\Crefname{theorem}{Theorem}{Theorems}
\newtheorem{lemma}[theorem]{Lemma}%
\crefname{lemma}{Lemma}{Lemmas}
\Crefname{lemma}{Lemma}{Lemmas}
\newtheorem{definition}[theorem]{Definition}%
\crefname{definition}{Definition}{Definitions}
\Crefname{definition}{Definition}{Definitions}
\newtheorem{remark}[theorem]{Remark}%
\crefname{remark}{Remark}{Remarks}
\Crefname{remark}{Remark}{Remarks}

% second block

\theoremstyle{nonumberplain}%
\theoremheaderfont{\bfseries}%
\theorembodyfont{\normalfont}%
\theoremseparator{.}
\theoremsymbol{\ensuremath{\Box}}% ensuremath -> sowohl im Text- als auch im Mathemodus wird das gewünschte Zeichen gesetzt

\newtheorem{proof}{Proof}%
\crefname{proof}{Proof}{Proofs}
\Crefname{proof}{Proof}{Proofs}
% ------------------------------------------------------------------------------
%\newenvironment{proof}[1][Proof]{\noindent\textbf{#1.} }{\ \rule{0.5em}{0.5em}}
%\textwidth =170mm
%\textheight=205mm
%\oddsidemargin=-8mm
%\evensidemargin=-8mm
\DeclareMathOperator*{\argmax}{arg\,max}
\DeclareMathOperator*{\argmin}{arg\,min}